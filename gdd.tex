% Options for packages loaded elsewhere
\PassOptionsToPackage{unicode}{hyperref}
\PassOptionsToPackage{hyphens}{url}
%
\documentclass[
]{article}
\author{}
\date{}

\usepackage{amsmath,amssymb}
\usepackage{lmodern}
\usepackage{iftex}
\ifPDFTeX
  \usepackage[T1]{fontenc}
  \usepackage[utf8]{inputenc}
  \usepackage{textcomp} % provide euro and other symbols
\else % if luatex or xetex
  \usepackage{unicode-math}
  \defaultfontfeatures{Scale=MatchLowercase}
  \defaultfontfeatures[\rmfamily]{Ligatures=TeX,Scale=1}
\fi
% Use upquote if available, for straight quotes in verbatim environments
\IfFileExists{upquote.sty}{\usepackage{upquote}}{}
\IfFileExists{microtype.sty}{% use microtype if available
  \usepackage[]{microtype}
  \UseMicrotypeSet[protrusion]{basicmath} % disable protrusion for tt fonts
}{}
\makeatletter
\@ifundefined{KOMAClassName}{% if non-KOMA class
  \IfFileExists{parskip.sty}{%
    \usepackage{parskip}
  }{% else
    \setlength{\parindent}{0pt}
    \setlength{\parskip}{6pt plus 2pt minus 1pt}}
}{% if KOMA class
  \KOMAoptions{parskip=half}}
\makeatother
\usepackage{xcolor}
\IfFileExists{xurl.sty}{\usepackage{xurl}}{} % add URL line breaks if available
\IfFileExists{bookmark.sty}{\usepackage{bookmark}}{\usepackage{hyperref}}
\hypersetup{
  hidelinks,
  pdfcreator={LaTeX via pandoc}}
\urlstyle{same} % disable monospaced font for URLs
\usepackage{longtable,booktabs,array}
\usepackage{calc} % for calculating minipage widths
% Correct order of tables after \paragraph or \subparagraph
\usepackage{etoolbox}
\makeatletter
\patchcmd\longtable{\par}{\if@noskipsec\mbox{}\fi\par}{}{}
\makeatother
% Allow footnotes in longtable head/foot
\IfFileExists{footnotehyper.sty}{\usepackage{footnotehyper}}{\usepackage{footnote}}
\makesavenoteenv{longtable}
\setlength{\emergencystretch}{3em} % prevent overfull lines
\providecommand{\tightlist}{%
  \setlength{\itemsep}{0pt}\setlength{\parskip}{0pt}}
\setcounter{secnumdepth}{-\maxdimen} % remove section numbering
\ifLuaTeX
  \usepackage{selnolig}  % disable illegal ligatures
\fi

\begin{document}

\begin{longtable}[]{@{}l@{}}
\toprule
\endhead
mainfont:``GFS Artemisia'' \\
\bottomrule
\end{longtable}

\hypertarget{ux3ccux3bdux3bfux3bcux3b1}{%
\section{Όνομα}\label{ux3ccux3bdux3bfux3bcux3b1}}

Το όνομα του παιχνιδιού μπορεί να αναφέρεται στο διαστημικό του
περιεχόμενο μιας που οι διαστημικές λέξεις είναι αρκετά δημοφιλείς ,
επίσης μπορεί να εμπεριέχει κάποια μεταφορά για την κατάσταση του παίκτη
ή του παιχνιδιού ,μερικές καλές εναλλακτικές:

\begin{itemize}
\tightlist
\item
  \textbf{Binary Stars Sparkles} (Το AI και ο καπετάνιος είναι δίδυμο
  που χρειάζεται ο ένας τον άλλον για να λάμψουν)
\item
  \textbf{Ex Umbra Runaway} (Πιο σκοτεινό ,ο παίκτης ξεφεύγει από τις
  ``σκιές'' aka από την έκλειψη)
\end{itemize}

\hypertarget{ux3c0ux3b5ux3c1ux3afux3bbux3b7ux3c8ux3b7-ux3c0ux3b1ux3b9ux3c7ux3bdux3b9ux3b4ux3b9ux3bfux3cd}{%
\section{Περίληψη
παιχνιδιού}\label{ux3c0ux3b5ux3c1ux3afux3bbux3b7ux3c8ux3b7-ux3c0ux3b1ux3b9ux3c7ux3bdux3b9ux3b4ux3b9ux3bfux3cd}}

Ένα 2d παιχνίδι με φωτεινά χρώματα , εύκολο χειρισμό και ανάλαφρη
αισθητική.\\
Το παιχνίδι θα έχει αρκετά στοιχεία idling αφού ο παίκτης θα παρατηρεί
τον καπετάνιο του πλοίου να συμμετέχει στις καθημερινές του
δραστηριότητες ωστόσο θα διακόπτεται απότομα από mini-games ταχύτητας
οπού το ai του διαστημόπλοιου (χαρακτήρας) θα καλείτε να βοηθήσει τον
καπετάνιο. Οι επιτυχίες και οι αποτυχίες στα παραπάνω mini games θα
αλλάζουν την διάθεση του καπετάνιου την οποία μετά ο παίκτης θα καλείτε
να κρίνει για συμπτωματολογία της κατάθλιψης.

\hypertarget{ux3b2ux3b1ux3c3ux3b9ux3baux3bfux3af-ux3c7ux3b1ux3c1ux3b1ux3baux3c4ux3aeux3c1ux3b5ux3c2}{%
\subsection{Βασικοί
χαρακτήρες}\label{ux3b2ux3b1ux3c3ux3b9ux3baux3bfux3af-ux3c7ux3b1ux3c1ux3b1ux3baux3c4ux3aeux3c1ux3b5ux3c2}}

\hypertarget{ux3b9ux3c3ux3c4ux3bfux3c1ux3afux3b1}{%
\subsection{Ιστορία}\label{ux3b9ux3c3ux3c4ux3bfux3c1ux3afux3b1}}

\hypertarget{gameplay}{%
\section{Gameplay}\label{gameplay}}

\hypertarget{ux3c3ux3c4ux3ccux3c7ux3bfux3b9}{%
\subsection{Στόχοι}\label{ux3c3ux3c4ux3ccux3c7ux3bfux3b9}}

\hypertarget{mechanics}{%
\subsection{Mechanics}\label{mechanics}}

Το παιχνίδι μπορεί να χωριστεί σε τρεις κατηγορίες από πλευρά mechanics

\hypertarget{examine}{%
\subsubsection{Examine}\label{examine}}

\begin{verbatim}
Στο τέλος τις μέρας το Ai καλείτε να χαρακτηρίσει τον καπετάνιο στο madrs-scale και μετα την αξιολόγηση του εμφανίζοντε στοιχεία για το πόσο πέτυχε στην 
κρίση του ή οχι.
\end{verbatim}

\hypertarget{acting}{%
\subsubsection{Acting}\label{acting}}

Το ποίο παιχνιδοποιημένο part, ο καπετάνιος θα αναλαμβάνει κάποια
καθημερινή εργασία και το ai θα καλείτε να το λύσει. Κάποια παραδείγματα
: - Ένωσε τα χαλασμένα καλώδια ίδιου χρώματος σε 10 δευτερόλεπτα για να
φτιάξεις την χαλασμένη μπαταρία ( point and click) - Βοήθα τον καπετάνιο
να κοιμηθεί πετυχαίνοντας τις νότες στο αυτόματο πιανάκι (guitar hero
style) - Βρές την κάρτα-κλειδί στο ακατάστατο δωμάτιο (hidden object )

\hypertarget{idling}{%
\subsubsection{Idling}\label{idling}}

\begin{verbatim}
Ο καπετάνιος κινείτε αυτόματα στα δωμάτια του διαστημόπλοιου για μικρά χρονικά διαστήματα μεταξυ των mini-games  ο παίκτης μπορεί να παρακολουθήσει αυτες τις 
κινήσεις και να συμπεράνει συμπτοματολογία απο αυτές
Μερικές εκφανσείς της συμπτοματολογίας :  
- Το ρομποτακι παρατηρεί μια γενικότερη ακαταστασία.
- Ο καπετάνιος κινείτε πιο αργά. 
- Δεν σηκώνει τα τηλέφωνα στους φίλους του αποφεύγει τη συναναστροφή με οποιωνδήποτε τρόπο.
- Αμελεί την προσωπική του υγιεινή ( δεν τρώει πρωινό,δεν πλένει τα δόντια του ,τα ρούχα του είναι βρώμικα ).
- Δεν κάνει tasks που του ανατίθενται από το AΙ  κάθεται αρκετά ακίνητος.
- Σκοτεινά δωμάτια,πολλές ώρες στο κρεβάτι χωρίς να κοιμάται.
- Κλειστά παράθυρα.
- Η μη έκφανση όλων των παραπάνω αποτελεί ένδειξη βελτίωσης της ψυχικής υγείας
- Η αποδοση σε κοματια του Acting phase  θα υποδηλώνουν 
\end{verbatim}

\hypertarget{miscsachievements}{%
\subsection{Miscs/Achievements}\label{miscsachievements}}

Όταν ο παίκτης πετυχαίνει στα mini-games κομμάτια του πλοίου
αναβαθμίζονται

\hypertarget{losing}{%
\subsection{Losing}\label{losing}}

Δεν θα υπάρχει σενάριο που ο παίκτης θα χάνει , ωστόσο αν ο παίκτης
χάνει σε πολλά mini-games ο χρόνος περάτωσης τους μπορεί να αυξάνεται
για διευκόλυνση του παίκτη.

\hypertarget{ux3b1ux3b9ux3c3ux3b8ux3b7ux3c4ux3b9ux3baux3aeux3b1ux3c4ux3bcux3ccux3c3ux3c6ux3b1ux3b9ux3c1ux3b1}{%
\subsection{Αισθητική/Ατμόσφαιρα}\label{ux3b1ux3b9ux3c3ux3b8ux3b7ux3c4ux3b9ux3baux3aeux3b1ux3c4ux3bcux3ccux3c3ux3c6ux3b1ux3b9ux3c1ux3b1}}

Τα γραφικά θα έχουν έντονα χρώματα σε pixel-art καθώς είναι συγκριτικά
εύκολο να φτιάξω ένα δικό μου pixel-art άμα δεν υπάρχει.

\hypertarget{ux3c4ux3b5ux3c7ux3bdux3b9ux3baux3ac-ux3c7ux3b1ux3c1ux3b1ux3baux3c4ux3b7ux3c1ux3b9ux3c3ux3c4ux3b9ux3baux3ac}{%
\subsection{Τεχνικά
χαρακτηριστικά}\label{ux3c4ux3b5ux3c7ux3bdux3b9ux3baux3ac-ux3c7ux3b1ux3c1ux3b1ux3baux3c4ux3b7ux3c1ux3b9ux3c3ux3c4ux3b9ux3baux3ac}}

Το παιχνίδι θα γραφεί σε godot(GDScript) πιθανός να υπάρχουν μερικά
hooks για c++ σε σημεία που απαιτούν παραπάνω κώδικα.

\end{document}
